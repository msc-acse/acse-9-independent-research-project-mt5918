{\color{blue}\textbf{\printdate{2019-07-21}}}

\bigbreak
\noindent\textbf{Parameters}\\
What are these parameters?:
\begin{enumerate}[topsep=0pt,itemsep=-1ex,partopsep=1ex,parsep=1ex,label=\Alph*)]
    \item \texttt{/ YD / YDC / ICFMTY}
    \item \texttt{/ YD / YDC / ICIATY}
\end{enumerate}

{\color{green}SOLVED}

\bigbreak
\noindent\textbf{HPC walk-in clinic \textbf{\printdate{2019-07-23}}}
        \begin{enumerate}[topsep=0pt,itemsep=-1ex,partopsep=1ex,parsep=1ex,label=\Alph*)]
            \item remove standard output and error from home directory
            \item make qsub.sh and qstat.sh working from local directory, for both ubuntu subsystem and git bash
            \item why do the paths work without variable but not with variables in bash, e.g. \$variable/qsub vs. bin/qsub
            \item scp secure copy does "setup2d/setup2d" -> I am confusing a subdirectory, how to copy directory into directory?
            \item .bashrc and .bash\_profile are broken in git bash, /esc/skil is empty;	where to get a fresh copy
            \item use paraview in hpc, should be all set up
            \item copy and paste with nano and vim
        \end{enumerate}
        
{\color{green}SOLVED}

\bigbreak
\noindent\textbf{Simulations}\\
Modify material parameters for geometry 1, e.g. penalty factor, buffer size. Studying the \texttt{2D} \texttt{Y} Code might give answers as to why the calculation breaks. An attempt is being made to write a \texttt{C++} program to generate a Y file. This could eventually then be integrated in the automation system to generate the Y file, automatically send it to the \texttt{HPC} system, calculate it and automatically analyse it using \texttt{Paraview}.
{\color{green}SOLVED}

\bigbreak
\noindent{\color{blue}\textbf{\printdate{2019-07-22}}} Feedback on approaches.
{\color{green}SOLVED}

\bigbreak
\noindent{\color{blue}\textbf{\printdate{2019-07-23}}} How to calculate the y file with windows batch file?
{\color{green}SOLVED}

\bigbreak
\noindent{\color{blue}\textbf{\printdate{2019-07-24}}} As the supervisors are away, there are no new revelations about what could have gone wrong with these simple simulations. An email was sent to the supervisors concerning questions about GiD:

\begin{enumerate}[topsep=0pt,itemsep=-1ex,partopsep=1ex,parsep=1ex,label=\Alph*)]
    \item What is the command line command to convert files in a GiD folder to a \texttt{Y} file?
    \item Do I need to create a new problem type instead of \texttt{B2D}?
    \item What is the \texttt{.dat} file and do I need it if I have y files?
\end{enumerate}
{\color{green}SOLVED}

\bigbreak
\noindent{\color{blue}\textbf{\printdate{2019-08-03}}}
There is an error with the node count in the \texttt{Y2D} program - the number of nodes of two rectangles drifting apart increases from $8$ to $20$, seemingly randomly. A non-buggy version of the code would be helpful, even if less developed. Otherwise time will be spent on fixing the errors.

{\color{green}SOLVED}: Supervisor said this is right, this is the creation of joint elements. Though why are they combined with the original elements, i.e. how to distinguish the two?

\bigbreak
What use is \texttt{GiD} on the \texttt{Linux} remote server? Command line arguments of calculating \texttt{.y} files from \texttt{.dat} files are not known to me (cannot be found in any manual), and a \texttt{GiD} \texttt{GUI} would surely not exist on \texttt{Linux}.

{\color{green}SOLVED}: Geraldo helped, gui works, {\color{blue}\textbf{\printdate{2019-08-06}}}: Santiago (\texttt{HPC} \texttt{RCS}) showed a way of enabling \texttt{X11} on \texttt{ese-pollux} remote server - thus making \texttt{putty} potentially superfluous. 

\bigbreak
\noindent{\color{blue}\textbf{\printdate{2019-08-03}}}
How to access the original elements and nodes - without the joint elements? Is this even necessary for the creation of the \texttt{vtk} files? It will be vital for the creation of joint elements between materials. {\color{green}SOLVED}: by the student. It is necessary to specify all points and elements, including the joint element ones in the output file. \texttt{paraview} knows what to do with them. No need to separate. 

\bigbreak
\noindent{\color{blue}\textbf{\printdate{2019-08-14}}}
Why is the encoding not working: What is the exact format of the vtk xml b64 encoded file?