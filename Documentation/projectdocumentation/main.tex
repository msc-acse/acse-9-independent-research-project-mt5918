%%%%%%%%%%%%%%%%%%%%%%%%%%%%%%%%%%%%%%%%%%%%%%%%%%%%%%%%%%%%%%%%%%%%%
% LaTeX Template: Project Titlepage Modified (v 0.1) by rcx
%
% Original Source: http://www.howtotex.com
% Date: February 2014
% 
% This is a title page template which be used for articles & reports.
% 
% This is the modified version of the original Latex template from
% aforementioned website.
% 
%%%%%%%%%%%%%%%%%%%%%%%%%%%%%%%%%%%%%%%%%%%%%%%%%%%%%%%%%%%%%%%%%%%%%%

\documentclass[12pt, UKenglish]{report}
\usepackage[a4paper]{geometry}
\usepackage[myheadings]{fullpage}
\usepackage{fancyhdr}
\usepackage{lastpage}
\usepackage{graphicx}
\usepackage{wrapfig}
\usepackage{subcaption}
\usepackage{setspace}
\usepackage{booktabs}
\usepackage[T1]{fontenc}
\usepackage[font=small, labelfont=bf]{caption}
\usepackage{fourier}
\usepackage[protrusion=true, expansion=true]{microtype}
\usepackage[UKenglish]{babel}
\usepackage{sectsty}
\usepackage{url, lipsum}
\usepackage{tgbonum}
\usepackage[colorlinks = true,
            linkcolor = blue,
            urlcolor  = blue,
            citecolor = blue,
            anchorcolor = blue]{hyperref}
\usepackage{xcolor}
\usepackage{isodate}
\usepackage{enumitem}
\usepackage{wrapfig}
\usepackage[backend=bibtex]{biblatex}
\usepackage{mdframed}
\usepackage{verbatimbox}

\usepackage{listings, color}    
\usepackage{textcomp}
\usepackage{fancyvrb}
\usepackage{verbatim}

\newcommand{\HRule}[1]{\rule{\linewidth}{#1}}
\onehalfspacing
\setcounter{tocdepth}{5}
\setcounter{secnumdepth}{5}

%-------------------------------------------------------------------------------
% HEADER & FOOTER
%-------------------------------------------------------------------------------
%\pagestyle{fancy}
%\fancyhf{}
%\setlength\headheight{15pt}
%\fancyhead[L]{Student ID: 1034511}
%\fancyhead[R]{Anglia Ruskin University}
%\fancyfoot[R]{Page \thepage\ of \pageref{LastPage}}
%-------------------------------------------------------------------------------
% TITLE PAGE
%-------------------------------------------------------------------------------

\begin{document}
{\fontfamily{cmr}\selectfont
\title{ \normalsize \textsc{}
		\\ [2.0cm]
		\HRule{0.5pt} \\
		\LARGE \textbf{\uppercase{Project Documentation}
		\HRule{2pt} \\ [0.5cm]
		\normalsize \vspace*{5\baselineskip}}
		}

\author{
		Michael Trapp\\
		Department of Earth Science and Engineering\\
		Imperial College London}

\maketitle
\tableofcontents
\newpage

%-------------------------------------------------------------------------------
% Section title formatting
\sectionfont{\scshape}
%-------------------------------------------------------------------------------

%-------------------------------------------------------------------------------
% BODY
%-------------------------------------------------------------------------------

\section{Highlights}
\bigbreak

\bigbreak
\noindent{\color{blue}\textbf{\printdate{2019-7-01}}}
There is an {\color{red} error} with the simulations. The \texttt{Y2D} code will compile, but the visualised output in \texttt{paraview} consists of random stripes and patterns.

\bigbreak
\noindent{\color{blue}\textbf{\printdate{2019-7-18}}}
The material parameters and simulation parameters are wrong. They need to be adjusted using formulas. This did not remove the visualisation errors, however.

\bigbreak
\noindent{\color{blue}\textbf{\printdate{2019-7-25}}} - {\color{blue}\textbf{\printdate{2019-7-28}}}
A \texttt{c++} file to generate the y file directly is now created, though still some bugs. However, the geometry still needs to be created separately independently. The approach is not scientifically relevant and was therefore abandoned.

\bigbreak
\noindent{\color{blue}\textbf{\printdate{2019-7-29}}}
Breakthrough during session with supervisor: The nodes for each element were listed in the wrong rotational order (counter-clockwise / clockwise), this lead to negative area, which lead to negative mass, which lead to code compilation errors. The error was perhaps produced by using \texttt{GiD} or \texttt{Y2D} code with the wrong operating system.

\bigbreak
\noindent{\color{blue}\textbf{\printdate{2019-8-02}}}
The simulations are running now, and showing good geometry. However, the body interactions are completely wrong. For example, a steel ball deforms on contact with the glass plies, instead of breaking it.

\bigbreak
\noindent{\color{blue}\textbf{\printdate{2019-8-07}}}\\
Legacy \texttt{vtk} file output works. \\
\texttt{VTK} \texttt{XML} file format output works, but only for \texttt{ascii} encoding.

\bigbreak
\noindent{\color{blue}\textbf{\printdate{2019-8-19}}}\\
\texttt{VTK} \texttt{XML} \texttt{b64} encoding looks relatively genuine, although there are still errors.


\newpage
\section{Project Specific Updates}
\par {\color{blue}\textbf{\printdate{2019-7-17}}}
Simple \texttt{2D} impact test were conducted. The geometry is illustrated in Fig. \ref{fig:initgeom}. However, these tests are not running.

\begin{wrapfigure}{R}{0.3\textwidth}
	\centering
	\includegraphics[width=0.3\textwidth]{initial_geometry.PNG}
	\caption{Simple test geometry ('geometry 1').}
	\label{fig:initgeom}
\end{wrapfigure}

\begin{figure}[!h]
	\centering
	\includegraphics[width=0.8\textwidth]{geometry2.PNG}
	\caption{A more sophisticated test geometry ('geometry 2').}
	\label{fig:geometry2}
\end{figure}

\bigbreak
\noindent{\color{blue}\textbf{\printdate{2019-7-18}}}
Meeting with supervisors. The parameters are wrong,as they need to be determined by use of formulas. For this purpose, it is necessary to introduce additional parameters (bonus parameters), such as total minimum element volume, minimum edge, critical time step of glass, steel, pvb, etc - these do not appear in the program UI. Using the formulas, the simple simulation is now running, provisionally using material for rock (as opposed to glass/steel/pvb).

\bigbreak
\noindent{\color{blue}\textbf{\printdate{2019-7-19}}}
Another geometry is adduced, see Fig. \ref{fig:geometry2}. Simulation using this new geometry, along with proper material data failed. This prompts the following conduction of systematical tests:

\begin{enumerate}[topsep=0pt,itemsep=-1ex,partopsep=1ex,parsep=1ex,label= {\color{blue}\textbf{test\arabic*}\,\,}]
    \item geometry 1, rock
    \item geometry 1, glass/steel
    \item geometry 2, fine mesh, rock
    \item geometry 1, glass
    \item geometry 1, steel
    \item geometry 1, glass, high penalty factor $p$
    \item geometry 1, steel, high penalty factor $p$
    \item geometry 2, fine mesh, glass
    \item geometry 2, fine mesh, steel
    \item geometry 2, fine mesh, pvb
    \item geometry 2, coarse mesh, rock
\end{enumerate}

\textbf{Conclusions from the tests}:

\begin{enumerate}[topsep=0pt,itemsep=-1ex,partopsep=1ex,parsep=1ex,label=\Alph*)]
    \item The issue is not with geometry 1, material rock or material steel.
    \item \textbf{Materials glass and pvb} needs be reconsidered, as tests with these materials have consistently not been successful.
    \item \textbf{Geometry 2} needs to perhaps be reconsidered. The fact that test 11 and test 3 fail implies that some parameters for geometry 2 are wrong.
\end{enumerate}

\bigbreak
{\color{blue}\textbf{\printdate{2019-7-22}}}

\newpage
\section{Things I Need Help with and my Plan to Resolve}
{\color{blue}\textbf{\printdate{2019-07-21}}}

\bigbreak
\noindent\textbf{Parameters}\\
What are these parameters?:
\begin{enumerate}[topsep=0pt,itemsep=-1ex,partopsep=1ex,parsep=1ex,label=\Alph*)]
    \item \texttt{/ YD / YDC / ICFMTY}
    \item \texttt{/ YD / YDC / ICIATY}
\end{enumerate}

{\color{green}SOLVED}

\bigbreak
\noindent\textbf{HPC walk-in clinic \textbf{\printdate{2019-07-23}}}
        \begin{enumerate}[topsep=0pt,itemsep=-1ex,partopsep=1ex,parsep=1ex,label=\Alph*)]
            \item remove standard output and error from home directory
            \item make qsub.sh and qstat.sh working from local directory, for both ubuntu subsystem and git bash
            \item why do the paths work without variable but not with variables in bash, e.g. \$variable/qsub vs. bin/qsub
            \item scp secure copy does "setup2d/setup2d" -> I am confusing a subdirectory, how to copy directory into directory?
            \item .bashrc and .bash\_profile are broken in git bash, /esc/skil is empty;	where to get a fresh copy
            \item use paraview in hpc, should be all set up
            \item copy and paste with nano and vim
        \end{enumerate}
        
{\color{green}SOLVED}

\bigbreak
\noindent\textbf{Simulations}\\
Modify material parameters for geometry 1, e.g. penalty factor, buffer size. Studying the \texttt{2D} \texttt{Y} Code might give answers as to why the calculation breaks. An attempt is being made to write a \texttt{C++} program to generate a Y file. This could eventually then be integrated in the automation system to generate the Y file, automatically send it to the \texttt{HPC} system, calculate it and automatically analyse it using \texttt{Paraview}.
{\color{green}SOLVED}

\bigbreak
\noindent{\color{blue}\textbf{\printdate{2019-07-22}}} Feedback on approaches.
{\color{green}SOLVED}

\bigbreak
\noindent{\color{blue}\textbf{\printdate{2019-07-23}}} How to calculate the y file with windows batch file?
{\color{green}SOLVED}

\bigbreak
\noindent{\color{blue}\textbf{\printdate{2019-07-24}}} As the supervisors are away, there are no new revelations about what could have gone wrong with these simple simulations. An email was sent to the supervisors concerning questions about GiD:

\begin{enumerate}[topsep=0pt,itemsep=-1ex,partopsep=1ex,parsep=1ex,label=\Alph*)]
    \item What is the command line command to convert files in a GiD folder to a \texttt{Y} file?
    \item Do I need to create a new problem type instead of \texttt{B2D}?
    \item What is the \texttt{.dat} file and do I need it if I have y files?
\end{enumerate}
{\color{green}SOLVED}

\bigbreak
\noindent{\color{blue}\textbf{\printdate{2019-08-03}}}
There is an error with the node count in the \texttt{Y2D} program - the number of nodes of two rectangles drifting apart increases from $8$ to $20$, seemingly randomly. A non-buggy version of the code would be helpful, even if less developed. Otherwise time will be spent on fixing the errors.

{\color{green}SOLVED}: Supervisor said this is right, this is the creation of joint elements. Though why are they combined with the original elements, i.e. how to distinguish the two?

\bigbreak
What use is \texttt{GiD} on the \texttt{Linux} remote server? Command line arguments of calculating \texttt{.y} files from \texttt{.dat} files are not known to me (cannot be found in any manual), and a \texttt{GiD} \texttt{GUI} would surely not exist on \texttt{Linux}.

{\color{green}SOLVED}: Geraldo helped, gui works, {\color{blue}\textbf{\printdate{2019-08-06}}}: Santiago (\texttt{HPC} \texttt{RCS}) showed a way of enabling \texttt{X11} on \texttt{ese-pollux} remote server - thus making \texttt{putty} potentially superfluous. 

\bigbreak
\noindent{\color{blue}\textbf{\printdate{2019-08-03}}}
How to access the original elements and nodes - without the joint elements? Is this even necessary for the creation of the \texttt{vtk} files? It will be vital for the creation of joint elements between materials. {\color{green}SOLVED}: by the student. It is necessary to specify all points and elements, including the joint element ones in the output file. \texttt{paraview} knows what to do with them. No need to separate. 

\bigbreak
\noindent{\color{blue}\textbf{\printdate{2019-08-14}}}
Why is the encoding not working: What is the exact format of the vtk xml b64 encoded file?

%\input{tinhw.tex}

%-------------------------------------------------------------------------------
% REFERENCES
%-------------------------------------------------------------------------------
\newpage
\section*{References}

%[2]John W. Eaton, David Bateman, Sren Hauberg, Rik Wehbring (2015). GNU
%Octave version 4.0.0 manual: a high-level interactive language for numer-
%ical computations. Available: http://www.gnu.org/software/octave/doc/
%interpreter/. 
}
\end{document}

%-------------------------------------------------------------------------------
% SNIPPETS
%-------------------------------------------------------------------------------

%\begin{figure}[!ht]
%	\centering
%	\includegraphics[width=0.8\textwidth]{file_name}
%	\caption{}
%	\centering
%	\label{label:file_name}
%\end{figure}

%\begin{figure}[!ht]
%	\centering
%	\includegraphics[width=0.8\textwidth]{graph}
%	\caption{Blood pressure ranges and associated level of hypertension (American Heart Association, 2013).}
%	\centering
%	\label{label:graph}
%\end{figure}

%\begin{wrapfigure}{r}{0.30\textwidth}
%	\vspace{-40pt}
%	\begin{center}
%		\includegraphics[width=0.29\textwidth]{file_name}
%	\end{center}
%	\vspace{-20pt}
%	\caption{}
%	\label{label:file_name}
%\end{wrapfigure}

%\begin{wrapfigure}{r}{0.45\textwidth}
%	\begin{center}
%		\includegraphics[width=0.29\textwidth]{manometer}
%	\end{center}
%	\caption{Aneroid sphygmomanometer with stethoscope (Medicalexpo, 2012).}
%	\label{label:manometer}
%\end{wrapfigure}

%\begin{table}[!ht]\footnotesize
%	\centering
%	\begin{tabular}{cccccc}
%	\toprule
%	\multicolumn{2}{c} {Pearson's correlation test} & \multicolumn{4}{c} {Independent t-test} \\
%	\midrule	
%	\multicolumn{2}{c} {Gender} & \multicolumn{2}{c} {Activity level} & \multicolumn{2}{c} {Gender} \\
%	\midrule
%	Males & Females & 1st level & 6th level & Males & Females \\
%	\midrule
%	\multicolumn{2}{c} {BMI vs. SP} & \multicolumn{2}{c} {Systolic pressure} & \multicolumn{2}{c} {Systolic Pressure} \\
%	\multicolumn{2}{c} {BMI vs. DP} & \multicolumn{2}{c} {Diastolic pressure} & \multicolumn{2}{c} {Diastolic pressure} \\
%	\multicolumn{2}{c} {BMI vs. MAP} & \multicolumn{2}{c} {MAP} & \multicolumn{2}{c} {MAP} \\
%	\multicolumn{2}{c} {W:H ratio vs. SP} & \multicolumn{2}{c} {BMI} & \multicolumn{2}{c} {BMI} \\
%	\multicolumn{2}{c} {W:H ratio vs. DP} & \multicolumn{2}{c} {W:H ratio} & \multicolumn{2}{c} {W:H ratio} \\
%	\multicolumn{2}{c} {W:H ratio vs. MAP} & \multicolumn{2}{c} {\% Body fat} & \multicolumn{2}{c} {\% Body fat} \\
%	\multicolumn{2}{c} {} & \multicolumn{2}{c} {Height} & \multicolumn{2}{c} {Height} \\
%	\multicolumn{2}{c} {} & \multicolumn{2}{c} {Weight} & \multicolumn{2}{c} {Weight} \\
%	\multicolumn{2}{c} {} & \multicolumn{2}{c} {Heart rate} & \multicolumn{2}{c} {Heart rate} \\
%	\bottomrule
%	\end{tabular}
%	\caption{Parameters that were analysed and related statistical test performed for current study. BMI - body mass index; SP - systolic pressure; DP - diastolic pressure; MAP - mean arterial pressure; W:H ratio - waist to hip ratio.}
%	\label{label:tests}
%\end{table}%\documentclass{article}
%\usepackage[utf8]{inputenc}

%\title{Weekly Report template}
%\author{gandhalijuvekar }
%\date{January 2019}

%\begin{document}

%\maketitle

%\section{Introduction}

%\end{document}


%\begin{enumerate}[topsep=0pt,itemsep=-1ex,partopsep=1ex,parsep=1ex,labe%l=\Alph*)]
%    \item total minimum element volume
%    \item total minimum edge
%    \item total real simulation time
%    \item critical time step glass
%    \item critical time step steel
%    \item critical time step pvb
%    \item critical time step rock
%    \item number of output files
%    \item projectile mesh size
%    \item ply mesh size
%    \item interlayer mesh size
%\end{enumerate}