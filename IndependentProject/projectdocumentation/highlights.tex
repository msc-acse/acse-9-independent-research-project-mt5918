\bigbreak

\bigbreak
\noindent{\color{blue}\textbf{\printdate{2019-7-01}}}
There is an {\color{red} error} with the simulations. The \texttt{Y2D} code will compile, but the visualised output in \texttt{paraview} consists of random stripes and patterns.

\bigbreak
\noindent{\color{blue}\textbf{\printdate{2019-7-18}}}
The material parameters and simulation parameters are wrong. They need to be adjusted using formulas. This did not remove the visualisation errors, however.

\bigbreak
\noindent{\color{blue}\textbf{\printdate{2019-7-25}}} - {\color{blue}\textbf{\printdate{2019-7-28}}}
A \texttt{c++} file to generate the y file directly is now created, though still some bugs. However, the geometry still needs to be created separately independently. The approach is not scientifically relevant and was therefore abandoned.

\bigbreak
\noindent{\color{blue}\textbf{\printdate{2019-7-29}}}
Breakthrough during session with supervisor: The nodes for each element were listed in the wrong rotational order (counter-clockwise / clockwise), this lead to negative area, which lead to negative mass, which lead to code compilation errors. The error was perhaps produced by using \texttt{GiD} or \texttt{Y2D} code with the wrong operating system.

\bigbreak
\noindent{\color{blue}\textbf{\printdate{2019-8-02}}}
The simulations are running now, and showing good geometry. However, the body interactions are completely wrong. For example, a steel ball deforms on contact with the glass plies, instead of breaking it.

\bigbreak
\noindent{\color{blue}\textbf{\printdate{2019-8-07}}}\\
Legacy \texttt{vtk} file output works. \\
\texttt{VTK} \texttt{XML} file format output works, but only for \texttt{ascii} encoding.

\bigbreak
\noindent{\color{blue}\textbf{\printdate{2019-8-19}}}\\
\texttt{VTK} \texttt{XML} \texttt{b64} encoding looks relatively genuine, although there are still errors.
