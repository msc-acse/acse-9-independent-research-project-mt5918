\section{Discussion}

\subsection{Simulation}

The simulation results are not realistic. The projectile of prescribed material steel locally deforms on impact on the glass, while the glass ply does not show any reaction to the impact.

\bigbreak
A likely cause might be implementation issues in the pre-processing softwares. The pre-processors \texttt{GID} and \texttt{AutoCAD} proved to be extremely unreliable. The input file prepared using \texttt{GID} contained contradictory definitions compared to the code. For example, the nodes which make up each element were defined in oppose rotational direction \texttt{GID}. It may be possible that importing geometries from \texttt{AutoCAD} produced this error. Another possibility may the that the preprocessor may produce different output on different operating systems. As code for these programs is not publicly available, further investigation is not possible.  

\bigbreak
Another plausible cause might be errors in the code. The code cannot be executed in a clean environment, such as the \texttt{Travis} environment in a \texttt{Github} repository. Specifically, the error occurs at the beginning of the program in line $733$ in file \lstinline{Yrd.c}, see List. \ref{lst:setlinebuferror}.

\begin{lstlisting}[language=C, caption=\lstinline{SETLINEBUF} Error in file \lstinline{Yrd.c},label=lst:setlinebuferror]
#define SETLINEBUF(fcheck) setvbuf((fcheck), NULL, _IONBF, 0); //definition
SETLINEBUF(ydc->fcheck); //line 733
\end{lstlisting}

The command \lstinline{setvbuf((fcheck), NULL, _IONBF, 0)} presumably assigns an empty buffer \lstinline{NULL} to some file \lstinline{ydc->fcheck}. However, the specifics and the purpose are unclear to the author.

\subsection{VTK Implementation}

The elementary implementation to generate valid \texttt{VTK} output files is successful.

Writing the output characters to a buffer instead of using \lstinline{putc} reduces the amount of times of invoking the \texttt{I/O} buffer and may save computational time. This alternative was discarded as it did not provide any significant advantage in terms of computational costs in this minimal example.

\bigbreak
The results can be applied to the \texttt{Y3D} code to simulate three-dimensional output.

\subsection{Future Work}

Future work includes the implementation of a compression algorithm to compress to the \texttt{VTK XML} files. This is indicated in the file by adding the option \lstinline[language=XML]{compressor="vtkZLibDataCompressor"} to the header \cite{Kit}. Bunge \cite{Bun09} provides a promising algorithm, but it remains to be revised, adjust and implemented. 

\bigbreak
As the pre-processor \texttt{GID} proved to be unpredictable on different operating systems and code is not available, it is of significant importance to replace this software with more reliable alternatives such as \texttt{Gmesh} \cite{Geu09}.

\bigbreak
Another important task is the debugging the original code. The first step should be to pass the \texttt{Github} repository build on a clean environment using a \texttt{Travis}. 

\bigbreak
Once the code is debugged, the next task is to add a feature to insert joint elements between elements of different materials.

\begin{enumerate}
    \item binary encoding (raw encoding)
    \item adjusting output of joint elements
\end{enumerate}